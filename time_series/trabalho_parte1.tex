% Options for packages loaded elsewhere
\PassOptionsToPackage{unicode}{hyperref}
\PassOptionsToPackage{hyphens}{url}
%
\documentclass[
]{article}
\title{Trabalho - cadeira de Séries Temporais}
\usepackage{etoolbox}
\makeatletter
\providecommand{\subtitle}[1]{% add subtitle to \maketitle
  \apptocmd{\@title}{\par {\large #1 \par}}{}{}
}
\makeatother
\subtitle{Parte 1}
\author{Pedro Miguel Sousa Magalhães}
\date{2021-11-12}

\usepackage{amsmath,amssymb}
\usepackage{lmodern}
\usepackage{iftex}
\ifPDFTeX
  \usepackage[T1]{fontenc}
  \usepackage[utf8]{inputenc}
  \usepackage{textcomp} % provide euro and other symbols
\else % if luatex or xetex
  \usepackage{unicode-math}
  \defaultfontfeatures{Scale=MatchLowercase}
  \defaultfontfeatures[\rmfamily]{Ligatures=TeX,Scale=1}
\fi
% Use upquote if available, for straight quotes in verbatim environments
\IfFileExists{upquote.sty}{\usepackage{upquote}}{}
\IfFileExists{microtype.sty}{% use microtype if available
  \usepackage[]{microtype}
  \UseMicrotypeSet[protrusion]{basicmath} % disable protrusion for tt fonts
}{}
\makeatletter
\@ifundefined{KOMAClassName}{% if non-KOMA class
  \IfFileExists{parskip.sty}{%
    \usepackage{parskip}
  }{% else
    \setlength{\parindent}{0pt}
    \setlength{\parskip}{6pt plus 2pt minus 1pt}}
}{% if KOMA class
  \KOMAoptions{parskip=half}}
\makeatother
\usepackage{xcolor}
\IfFileExists{xurl.sty}{\usepackage{xurl}}{} % add URL line breaks if available
\IfFileExists{bookmark.sty}{\usepackage{bookmark}}{\usepackage{hyperref}}
\hypersetup{
  pdftitle={Trabalho - cadeira de Séries Temporais},
  pdfauthor={Pedro Miguel Sousa Magalhães},
  hidelinks,
  pdfcreator={LaTeX via pandoc}}
\urlstyle{same} % disable monospaced font for URLs
\usepackage[margin=1in]{geometry}
\usepackage{longtable,booktabs,array}
\usepackage{calc} % for calculating minipage widths
% Correct order of tables after \paragraph or \subparagraph
\usepackage{etoolbox}
\makeatletter
\patchcmd\longtable{\par}{\if@noskipsec\mbox{}\fi\par}{}{}
\makeatother
% Allow footnotes in longtable head/foot
\IfFileExists{footnotehyper.sty}{\usepackage{footnotehyper}}{\usepackage{footnote}}
\makesavenoteenv{longtable}
\usepackage{graphicx}
\makeatletter
\def\maxwidth{\ifdim\Gin@nat@width>\linewidth\linewidth\else\Gin@nat@width\fi}
\def\maxheight{\ifdim\Gin@nat@height>\textheight\textheight\else\Gin@nat@height\fi}
\makeatother
% Scale images if necessary, so that they will not overflow the page
% margins by default, and it is still possible to overwrite the defaults
% using explicit options in \includegraphics[width, height, ...]{}
\setkeys{Gin}{width=\maxwidth,height=\maxheight,keepaspectratio}
% Set default figure placement to htbp
\makeatletter
\def\fps@figure{htbp}
\makeatother
\setlength{\emergencystretch}{3em} % prevent overfull lines
\providecommand{\tightlist}{%
  \setlength{\itemsep}{0pt}\setlength{\parskip}{0pt}}
\setcounter{secnumdepth}{-\maxdimen} % remove section numbering
\ifLuaTeX
  \usepackage{selnolig}  % disable illegal ligatures
\fi

\begin{document}
\maketitle

\hypertarget{question}{%
\subsection{Question:}\label{question}}

Consider the time series

\[ x_t = \beta_0 + \beta_1t + \omega_t\]

Where \(\beta_0\) and \(\beta_1\) are regression coefficients, and
\(\omega_t\) is a white noise process with variance \(\omega_t^2\)

\begin{enumerate}
\def\labelenumi{(\alph{enumi})}
\tightlist
\item
  Determine whether \(x_t\) is stationary
\item
  Show that the process \(y_t = x_t - x_{t-1}\) is stationary
\item
  Show that the mean of the two-sided moving average
\end{enumerate}

\[ \nu_t = \frac{1}{3}(x_{t-1} + x_t + x_{t+1})\]

is \(\beta_0 + \beta_1t\)

\hypertarget{answer}{%
\subsection{Answer:}\label{answer}}

\textbf{Question a):}

A time series is weak stationarity (henceafter just stationarity) when
both conditions are met:

\begin{enumerate}
\def\labelenumi{\roman{enumi})}
\tightlist
\item
  The expected value \(E(x_t)\) is constant and non dependent on the
  value of \(t\)
\item
  The autocovariance between two points of the time series in time
  \((x_t, x_s)\) depends only on the difference time between then
  \((t-s)\)
\end{enumerate}

Considering the properties of the expected value specially given that
the expected value of a constant value (scalar) is the value itselfe and
white noise as expected value equal to zero, we can calculate the
expected value of the above time series as follows:

\[\mu_{x_t} = E(x_t) = E(\beta_0 + \beta_1t + \omega_t) = E(\beta_0) + E(\beta_1t)+E(\omega_t) = \beta_0 + \beta_1t \]
From the above expression we can conclude that the expected value is
dependent of \(t\) and therefore violates one of the condition for
stationarity \textbf{therefore the answer to (a) is that the time series
is not stationary.}

\begin{longtable}[]{@{}
  >{\raggedright\arraybackslash}p{(\columnwidth - 0\tabcolsep) * \real{0.06}}@{}}
\toprule
\endhead
\textbf{Question b):} \\
Following the above definition of stationarity we follow a similar test
for \(y_t\) (or lag 1). But first we can simplify the \(y_t\) into the
following: \\
\(y_t = (\beta_0 + \beta_1t + \omega_t) - (\beta_0 + \beta_1(t-1)+ \omega_{t-1}) = \beta_1(t - t + 1) + \omega_t - \omega_{t-1} = \beta_1 + \omega_t - \omega_{t-1}\) \\
Calculating the expected value we get: \\
\(\mu_{y_t} = E(\beta_1 + \omega_t - \omega_{t-1}) = \beta_1\) \\
Since the result is a constant independent of t we can conclude that the
\textbf{first condition of stationarity is met}. \\
As for the second condition considering \(h\) as the time difference
between two observation of the distribution \(y\) and given the results
obtained as expected value we can conclude the following \\
\(\gamma_y(h) = COV(y_{t+h}, y_t) = E[(y_{t+h} - \mu_{y,t+h})(y_t-\mu_{y,t})] = E[(w_{t+h}-w_{t+h-1})(w_t-w_{t-1})]\)
Given that \(E(w) = 0 \implies \gamma_y(h) = 0\) and when \$h \ne 0 \$
then: \\
\(\gamma_y(h) = E(w_t^2 - w_{t-1} - w_tw_{t-1}+w_t^2) = E[(w_t-w_{t-1})^2] = \sigma_w^2\) \\
Meanting that it depends on \(h\) and hence proves that \(y_t\) complies
with \textbf{both conditions for stationarity.} \\
\bottomrule
\end{longtable}

\begin{center}\rule{0.5\linewidth}{0.5pt}\end{center}

\textbf{Question c):}

\[E(v_1) = \beta_0 + \beta_1t \implies E(v_1) = E[\frac{1}{3}(x_{t-1} + x_t + x_{t+1})] = \frac{1}{3}E(x_{t-1}) + \frac{1}{3}E(x_t) + \frac{1}{3}E(x_{t+1}) = \\ \frac{1}{3}[\beta_0 + \beta_1(t-1)+\beta_0+\beta_1t+\beta-0+\beta_1(t+1)] = \frac{3}{3}(\beta_0 + \beta_1t) = \beta_0 + \beta_1t\]

\end{document}
